\documentclass[hyperref, UTF8, CJK]{beamer}

\usetheme[ColorDisplay=BSblue]{scu}

\usepackage{multicol,multirow}

\title[五连鞭的运气要领]{马掌门讲五连鞭的运气要领}
\subtitle{混元形翼太极门弟子的必修课}
\author[掌门人, 首席大弟子]{马老卷\inst{1}\inst{a} \and 马小卷\inst{2}\inst{b}}
\institute{%
  \inst{1} 混元形翼太极门
  \vspace*{-6pt} \and
  \inst{2} ~Management Science, Business School, Sichuan University
  \vspace*{-6pt} \and
  \inst{a} ~\textit{MaLJFake@taichi.hunyuan} ~\inst{b} ~\textit{MaXJFake@scu.edu.cn}
}
\date{2020 年 11 月 15 日}

\begin{document}
\section{蓝色主题}
\subsection{宝石蓝}
\begin{frame}{三个字}
  接~化~发
  \begin{itemize}
    \item 接~化~发
  \end{itemize}
  \begin{enumerate}
    \item 接~化~发
  \end{enumerate}
  \begin{scutheorem}{切比雪夫大数率}
		对独立随机变量序列$\{X_k\}$, 若$E(X_k)$, $D(X_k)$都存在, $k=1,2,\cdots$, 且有常数$C$, 使得$D(X_k)\leq C$, $k=1,2,\cdots$, 则有
		\begin{equation}
		\dfrac{1}{n} \sum_{k=1}^{n} X_k - \dfrac{1}{n} \sum_{k=1}^{n} E(X_k) \stackrel{\;P\;}{\longrightarrow} 0
		\end{equation}
	\end{scutheorem}
\end{frame}

\begin{frame}
  
\end{frame}
\end{document}